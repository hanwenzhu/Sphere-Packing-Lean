\subsection{Scaling Sphere Packings}

Given that the problem involves the \emph{arrangement} of balls in space, it is intuitive not to worry about the radius of the balls (so long as they are all equal to each other). However, Definition~\ref{SpherePacking.balls} involves a choice of separation radius. In principle, we would want two sphere packing configurations that differ only in separation radii to `encode the same information'. In this brief subsection, we will describe how to change the separation radius of a sphere packing by \emph{scaling} the packing by a positive real number and prove that this does not affect its density. This will give us the freedom to choose any separation radius we like when attempting to define the optimal sphere packing in $\R^d$.

\inputleannode{SpherePacking.scale}

\inputleannode{SpherePacking.scale_finiteDensity}


\inputleannode{SpherePacking.scale_density}


Therefore, as expected, we do not need to worry about the separation radius when constructing sphere packings. This will be useful when we attempt to construct the optimal sphere packing in $\R^8$---and even more so when attempting to \emph{formalise} this construction---because the underlying structure of the packing is given by a set known as the $E_8$ lattice, which has separation radius $\sqrt{2}$.

We can also use Lemma~\ref{SpherePacking.scale_density} to simplify the computation of the sphere packing constant by taking the supremum not over \emph{all} sphere packings but only over those with density $1$.

\inputleannode{SpherePacking.constant_eq_constant_normalized}