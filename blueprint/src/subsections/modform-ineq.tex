\section{Proof of Theorem \ref{thm:g}}\label{sec: g}
Our proof of the Theorem~\ref{thm:g} relies on the following two inequalities for modular objects.
\begin{proposition}\label{prop:ineqA}\uses{lemma:ineqABnew-equiv, lemma:F-G-phi-psi-identities, lemma:F-G-pos, cor:ineqAnew}
Consider the function $A:(0,\infty)\to\C$ defined as
\begin{equation}\label{eqn:defA}
A(t):=-t^2\phi_0(i/t)-\frac{36}{\pi^2}\,\psi_I(it).
\end{equation}
Then
\begin{equation}\label{eqn:ineqA}
    A(t) < 0
\end{equation}
for all $t > 0$.
\end{proposition}

\begin{proposition}\label{prop:ineqB}\uses{lemma:ineqABnew-equiv, lemma:F-G-phi-psi-identities, cor:ineqBnew}
Consider the function $B:(0,\infty)\to\C$ defined as
\begin{equation}\label{eqn:defB}
    B(t) := -t^2\phi_0(i/t)+\frac{36}{\pi^2}\,\psi_I(it)
\end{equation}
Then
\begin{equation}\label{eqn:ineqB}
    B(t) > 0
\end{equation}
for all $t > 0$.
\end{proposition}

Here we formalize the proof of the inequalities by Lee \cite{Lee}.
First, we can rewrite the inequality in \ref{prop:ineqA} as follows.

\begin{definition}\label{def:FG-definition}\uses{E₂_eq, ModularForm.eisensteinSeries_MF, def:H2-H3-H4}
Define two (quasi) modular forms as
\begin{align}
    F(z) &= (E_2(z) E_4(z) - E_6(z))^2 \label{eqn:defF} \\
    G(z) &= H_2(z)^{3} (2 H_{2}(z)^{2} + 5 H_{2}(z) H_{4}(z) + 5 H_{4}(z)^{2}). \label{eqn:defG}
\end{align}
\end{definition}

\begin{lemma}\label{lemma:F-G-phi-psi-identities}\uses{def:FG-definition, lemma:psiS-new}
We have
\begin{align}
    \phi_0 &= \frac{F}{\Delta} \label{eqn:phi0-F} \\
    \psi_S &= -\frac{1}{2} \frac{G}{\Delta}\label{eqn:psiS-G}
\end{align}
\end{lemma}
\begin{proof}
\eqref{eqn:phi0-F} is clear.
\eqref{eqn:psiS-G} is already shown in Lemma \ref{lemma:psiS-new}.
\end{proof}

\begin{lemma}\label{lemma:ineqABnew-equiv}\uses{lemma:F-G-phi-psi-identities, def:psiI-psiT-psiS, Delta_im_line}
Inequality \eqref{eqn:ineqA} and \eqref{eqn:ineqB} are equivalent to
\begin{align}
    F(it) + \frac{18}{\pi^2} G(it) > 0 \label{eqn:ineqAnew} \\
    F(it) - \frac{18}{\pi^2} G(it) > 0 \label{eqn:ineqBnew}
\end{align}
respectively.
\end{lemma}
\begin{proof}
By \eqref{eqn:psiS-define},
\begin{equation}
    \psi_I(it) = (\psi_S|_{-2}S)(it) = (it)^{2}\psi_S\left(-\frac{1}{it}\right) = -t^2 \psi_S\left(\frac{i}{t}\right).
\end{equation}
Combined with Lemma \ref{lemma:F-G-phi-psi-identities} we can rewrite \eqref{eqn:ineqA} as
\begin{equation}
    A(t) = -t^2 \phi_0\left(\frac{i}{t}\right) + \frac{36}{\pi^2} \psi_S\left(\frac{i}{t}\right) < 0 \Leftrightarrow \frac{F(it)}{\Delta(it)} + \frac{18}{\pi^2} \frac{G(it)}{\Delta(it)} > 0
\end{equation}
for $t > 0$, which is equivalent to \eqref{eqn:ineqAnew} by Corollary \ref{Delta_im_line}.
Equivalences of \eqref{eqn:ineqB} and \eqref{eqn:ineqBnew} follows similarly; just change the sign.
\end{proof}


Now, the first inequality \eqref{eqn:ineqAnew} follows from the positivity of each $F(it)$ and $G(it)$.

\begin{lemma}\label{lemma:F-G-pos}\uses{ramanujan_E₂, cor:theta-pos}
For all $t > 0$, we have $F(it) > 0$ and $G(it) > 0$.
\end{lemma}
\begin{proof}
By Ramanujan's identity \eqref{eqn:DE4}, we have $F(z) = 9 E_4'(z)^2$ and
\begin{equation}
    F(it) = 9E_4'(it)^2 = 9 \left(240\sum_{n \geq 1} n \sigma_3(n) e^{-2 \pi n t} \right)^{2} > 0.
\end{equation}
$G(it) > 0$ follows from positivity of $H_2(it)$ and $H_4(it)$ (Lemma \ref{cor:theta-pos}).
\end{proof}

\begin{corollary}\label{cor:ineqAnew}\uses{lemma:F-G-pos}
\eqref{eqn:ineqAnew} holds.
\end{corollary}
\begin{proof}
This directly follows from Lemma \ref{lemma:F-G-pos}.
\end{proof}

To prove the second inequality \eqref{eqn:ineqBnew}, we need some identities satisfied by $F$ and $G$.
\begin{lemma}\label{lemma:F-G-de}\uses{ramanujan_E₂, serre_D_mul, prop:theta-der, lemma:lv1-lv2-identities}
$F$ and $G$ satisfy the following differential equations:
\begin{align}
    \partial_{12}\partial_{10} F - \frac{5}{6} E_{4} F &= 7200 \Delta (-E_{2}') \label{eqn:ddf} \\
    \partial_{12}\partial_{10} G - \frac{5}{6} E_{4} G &= -640 \Delta H_{2} \label{eqn:ddg}
\end{align}
\end{lemma}
\begin{proof}
Both can be shown by direct computations.
By Ramanujan's identities (Theorem \ref{ramanujan_E₂}) and the product rule of Serre derivatives (Theorem \ref{serre_D_mul}), we have
\begin{align}
    \partial_{5} (E_2 E_4 - E_6) &= (E_2 E_4 - E_6)' - \frac{5}{12} E_2 (E_2 E_4 - E_6) \\
    &= \frac{E_2^2 - E_4}{12} \cdot E_4 + E_2 \cdot \frac{E_2 E_4 - E_6}{3} - \frac{E_2 E_6 - E_4^2}{2} - \frac{5}{12}E_2 (E_2 E_4 - E_6) \\
    &= -\frac{5}{12} (E_2 E_6 - E_4^2) \label{eqn:S5} \\
    \partial_{7} (E_2 E_6 - E_4^2) &= (E_2 E_6 - E_4^2)' - \frac{7}{12} E_2 (E_2 E_6 - E_4^2) \\
    &= \frac{E_2^2 - E_4}{12} \cdot E_6 + E_2 \cdot \frac{E_2 E_6 - E_4^2}{2} - 2 E_4 \cdot \frac{E_2 E_4 - E_6}{3} - \frac{7}{12} E_2 (E_2 E_6 - E_4^2) \\
    &= -\frac{7}{12} E_4 (E_2 E_4 - E_6) \label{eqn:S7}
\end{align}
and using these we can compute
\begin{align}
    \partial_{10} F &= \partial_{10} (E_2 E_4 - E_6)^2 \\
    &= 2 (E_2 E_4 - E_6) \partial_{5} (E_2 E_4 - E_6) \\
    &= -\frac{6}{5} (E_2 E_4 - E_6) (E_2 E_6 - E_4^2), \\
    \partial_{12}\partial_{10} F &= -\frac{5}{6} \partial_{12} ((E_2 E_4 - E_6) (E_2 E_6 - E_4)) \\
    &= -\frac{5}{6} (\partial_{5}(E_2 E_4 - E_6)) (E_2 E_6 - E_4^2) - \frac{5}{6} (E_2 E_4 - E_6) (\partial_{7} (E_2 E_6 - E_4)) \\
    &= \frac{25}{72} (E_2 E_6 - E_4^2)^2 + \frac{35}{72} E_4 (E_2 E_4 - E_6)^2, \\
    \partial_{12}\partial_{10}F - \frac{5}{6} E_4 F &= \frac{25}{72}(E_2 E_6 - E_4^2)^2 + \frac{35}{72} E_4 (E_2 E_4 - E_6)^2 - \frac{5}{6} E_4 (E_2 E_4 - E_6)^2 \\
    &= \frac{25}{72} ((E_2 E_6 - E_4^2)^2 - E_4 (E_2 E_4 - E_6)^2) \\
    &= \frac{25}{72} (- E_2^2 E_4^3 + E_2^2 E_6^2 + E_4^4 - E_4 E_6^3) \\
    &= -\frac{25}{72} (E_4^3 - E_6^2) (E_2^2 - E_4) \\
    &= 7200 \cdot \frac{E_4^3 - E_6^2}{1728} \cdot \frac{-E_2^2 + E_4}{12}\\
    &= 7200 \Delta (-E_2')
\end{align}
which proves \eqref{eqn:ddf}.
Similarly, \eqref{eqn:ddg} can be proved using Proposition \ref{prop:theta-der} and Lemma \ref{lemma:lv1-lv2-identities}.
\end{proof}

\begin{corollary}\label{cor:F-G-de}\uses{lemma:F-G-de, Delta_im_line, cor:theta-pos, E₂_eq}
\eqref{eqn:ddf} (resp. \eqref{eqn:ddg}) is positive (resp. negative) on the (positive) imaginary axis.
\end{corollary}
\begin{proof}
From \eqref{eqn:E2} and Lemma \ref{Delta_im_line},
\begin{equation}
    7200 (-E_2'(it)) \Delta(it) = 7200 \cdot 24 \left(\sum_{n \ge 1} n \sigma_1(n) e^{-2 \pi n t}\right) \cdot \Delta(it) > 0. \notag
\end{equation}
Negativity of \eqref{eqn:ddg}, i.e. $-640 \Delta(it) H_2(it) < 0$ follows from Corollary \ref{cor:theta-pos} and \ref{Delta_im_line}.
\end{proof}


The second inequality \eqref{eqn:ineqBnew} follows from the following two observations.
Since $G(it) > 0$ for all $t > 0$, we can define the quotient
\begin{equation}\label{eqn:Q}
    Q(t) := \frac{F(it)}{G(it)}
\end{equation}
as a function on $(0, \infty)$.

\begin{lemma}\label{lemma:Qlim}\uses{E₂_transform, EisensteinSeries.eisensteinSeries_SIF, H₂_T_action}
We have
\begin{equation}\label{eqn:Qlim}
    \lim_{t \to 0^+} Q(t) = \frac{18}{\pi^2}.
\end{equation}
\end{lemma}
\begin{proof}
We have
\begin{equation}
    \lim_{t \to 0^+} Q(t) = \lim_{t \to 0^+} \frac{F(it)}{G(it)} = \lim_{t \to \infty} \frac{F(i/t)}{G(i/t)}.
\end{equation}
By using the transformation laws of Eisenstein series \eqref{eqn:E2-S-transform}, \eqref{eqn:Ek-trans-S} (for $k = 4, 6$) and the thetanull functions, \eqref{eqn:H2-transform-S}, \eqref{eqn:H4-transform-S}, we get
\begin{align}
    F\left(\frac{i}{t}\right) &= t^{12} F(it) - \frac{12t^{11}}{\pi} (E_2(it)E_4(it) - E_6(it))E_4(it) + \frac{36t^{10}}{\pi^2}E_4(it)^2, \\
    G\left(\frac{i}{t}\right) &= t^{10} H_{4}(it)^{3}(2H_{4}(it)^{2} + 5 H_{4}(it)H_{2}(it) + 5 H_{2}(it)^{2}).
\end{align}
Since $F$, $E_2 E_4 - E_6$ and $H_2$ are cusp forms, we have $\lim_{t \to \infty}t^k A(it) = 0$ when $A(z)$ is one of these forms and $k \geq 0$.
From $\lim_{t \to \infty} E_4(it) = 1 = \lim_{t \to \infty}H_{4}(it)$, we get
\begin{align}
    \lim_{t \to \infty} \frac{F(i/t)}{G(i/t)}
    &= \lim_{t \to \infty} \frac{t^{12} F(it) - \frac{12t^{11}}{\pi} (E_2(it)E_4(it) - E_6(it))E_4(it) + \frac{36t^{10}}{\pi^2}E_4(it)^2}{t^{10} H_{4}(it)^{3}(2H_{4}(it)^{2} + 5 H_{4}(it)H_{2}(it) + 5 H_{2}(it)^{2})} \\
    &= \lim_{t \to \infty} \frac{t^{2} F(it) - \frac{12t}{\pi} (E_2(it)E_4(it) - E_6(it))E_4(it) + \frac{36}{\pi^2}E_4(it)^2}{H_{4}(it)^{3}(2H_{4}(it)^{2} + 5 H_{4}(it)H_{2}(it) + 5 H_{2}(it)^{2})} \\
    &= \frac{18}{\pi^2}.
\end{align}
\end{proof}

\begin{proposition}\label{prop:Qdec}\uses{ramanujan_E₂, serre_D_mul, cor:F-G-de, thm:anti-serre-der-pos}
The function $t \mapsto Q(t)$ is monotone decreasing.
\end{proposition}

\begin{proof}
It is enough to show that
\begin{align}
    \frac{\dd}{\dd t} \left(\frac{F(it)}{G(it)}\right) < 0 &\Leftrightarrow (- 2\pi) \frac{F'(it)G(it) - F(it) G'(it)}{G(it)^{2}} < 0 \\
    &\Leftrightarrow F'(it) G(it) - F(it) G'(it) > 0 \\
    &\Leftrightarrow (\partial_{10}F)(it) G(it) - F(it) (\partial_{10}G)(it) > 0.
\end{align}
Let $\mathcal{L}_{1, 0} := (\partial_{10}F) G - F (\partial_{10} G)$.
Then its Fourier expansion starts with
\begin{equation}
    \mathcal{L}_{1, 0} = 5308416000 q^{\frac{7}{2}} + O(q^{\frac{9}{2}}) \notag
\end{equation}
and its Serre derivative $\partial_{22} \mathcal{L}_{1, 0}$ is positive by Corollary \ref{cor:F-G-de}:
\begin{align}
    \partial_{22} \mathcal{L}_{1, 0} = (\partial_{12} \partial_{10} F) G - F (\partial_{12}\partial_{10} G)
    = \Delta (7200 (-E_{2}') G + 640 H_2 F) > 0.
\end{align}
Hence $\mathcal{L}_{1, 0}(it) > 0$ by Theorem \ref{thm:anti-serre-der-pos}, and the monotonicity follows.
\end{proof}


\begin{corollary}\label{cor:ineqBnew} \uses{lemma:Qlim, prop:Qdec, lemma:F-G-pos}
\eqref{eqn:ineqBnew} holds.
\end{corollary}
\begin{proof}
\begin{equation}
    \frac{F(it)}{G(it)} = Q(t) < \lim_{u \to 0^+} Q(u) = \frac{18}{\pi^2}
\end{equation}
and by Lemma \ref{lemma:F-G-pos}, \eqref{eqn:ineqBnew} follows.
\end{proof}


Finally, we are ready to prove Theorem~\ref{thm:g}.
\begin{theorem}\label{thm:g1}
\uses{MagicFunction.a.Fourier.eig_a, MagicFunction.b.Fourier.eig_b, prop:a-double-zeros, prop:b-double-zeros, prop:ineqA, prop:ineqB, MagicFunction.a.SpecialValues.a_zero, MagicFunction.b.SpecialValues.b_zero}
The function
$$g(x):=\frac{\pi\,i}{8640}a(x)+\frac{i}{240\pi}\,b(x)$$
satisfies conditions \eqref{eqn:g1}--\eqref{eqn:g3}.
\end{theorem}
\begin{proof}
First, we prove that \eqref{eqn:g1} holds. By Propositions~\ref{prop:a-double-zeros} and \ref{prop:b-double-zeros} we know that for $r>\sqrt{2}$
\begin{equation}\label{eqn:g A} g(r)=\frac{\pi}{2160}\,\sin(\pi r^2/2)^2\,\int\limits_0^\infty A(t)\,e^{-\pi r^2 t}\,dt\end{equation}
where $$A(t)=-t^2\phi_0(i/t)-\frac{36}{\pi^2}\,\psi_I(it).$$
from the Proposition~\ref{prop:ineqA} we know that $A(t)<0\quad\mbox{for}\;t\in(0,\infty).$
Therefore identity \eqref{eqn:g A} implies \eqref{eqn:g1}.

Next, we prove \eqref{eqn:g2}. By Propositions~\ref{prop:a-another-integral} and~\ref{prop:b-another-integral} we know that for $r>0$
\begin{equation}\label{eqn:g B} \widehat{g}(r)=\frac{\pi}{2160}\,\sin(\pi r^2/2)^2\,\int\limits_0^\infty B(t)\,e^{-\pi r^2 t}\,dt\end{equation}
where $$B(t)=-t^2\phi_0(i/t)+\frac{36}{\pi^2}\,\psi_I(it).$$


Finally, the property \eqref{eqn:g3} readily follows from Proposition~\ref{MagicFunction.a.SpecialValues.a_zero} and Proposition~\ref{MagicFunction.b.SpecialValues.b_zero}.
This finishes the proof of Theorems~\ref{thm:g1} and~\ref{thm:g}.
\end{proof}